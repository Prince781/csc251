\documentclass[a4paper,11pt]{article}
\usepackage[T1]{fontenc}
\usepackage[utf8]{inputenc}
\usepackage{lmodern}

\title{CSC 2/451 - Project 1}
\author{
Yufei Du \\
\texttt{<ydu14@ur.rochester.edu>}
\and
Princeton Ferro \\
\texttt{<pferro@u.rochester.edu>}
}

\begin{document}

\maketitle

\section{iCache}
Implemented in \texttt{verilog/ICACHE.v} as a 32 KB read-only direct-mapped cache with 32 B block size. The current PC is used to index into the cache. $ 32KB / 32 B$ is 1024, so the index is 10 bits. Each instruction is 4-byte aligned, so the lower 2 bits can be ignored, while the 3 bits after that will be the number of instructions into the cache entry. Therefore, the lower 5 bits are the offset. This leaves us with $32-10-5 =$ 17 bits for the tag.

The 10-cycle miss penalty is achieved by an internal counter that, when nonzero and less than 10, stalls the IF and ID stages. IF and ID don't proceed if the \texttt{valid} from the iCache is negative.

Additionally, \texttt{sim\_main.cpp} was modified at lines 823 and 827 to fetch on aligned addresses.

\section{dCache}
Implemented in \texttt{verilog/DCACHE.v}.

\end{document}
